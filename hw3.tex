\documentclass[12pt,letterpaper, onecolumn]{exam}
\usepackage{amsmath}
\usepackage{amssymb}
\usepackage[lmargin=71pt, tmargin=1.2in]{geometry}
\usepackage{graphicx}%For centering solution box
\lhead{IND ENG 221\\}
\rhead{Kartikeya Sharma\\}
% \chead{\hline} % Un-comment to draw line below header
\thispagestyle{empty}   %For removing header/footer from page 1

\begin{document}

\begingroup  
    \centering
    \LARGE IND ENG 221\\
    \LARGE HW 3\\[0.5em]
    \large \today\\[0.5em]
    \large Kartikeya Sharma\par
    \large SID: 3037376860\par
\endgroup
\rule{\textwidth}{0.4pt}
\pointsdroppedatright   %Self-explanatory
\printanswers
\renewcommand{\solutiontitle}{\noindent\textbf{Ans:}\enspace}   %Replace "Ans:" with starting keyword in solution box

\begin{questions}

    \question A \$100 million interest rate swap has a remaining life of 10 months. Under the terms
of the swap, six-month LIBOR is exchanged for 4\% per annum (compounded
semiannually). Six-month LIBOR forward rates for all maturities are 3\% (with
semiannual compounding). The six-month LIBOR rate was 2.4\% two months ago. OIS
rates for all maturities are 2.7\% with continuous compounding. What is the current value
of the swap to the party paying floating? What is the value to the party paying fixed?
(Hull 7.12)
    
    \begin{solution}

    \subsection*{Step 1: Present Value of the Fixed Leg}
The fixed rate is 4\% annually, paid semiannually. The payment dates are in 6 months ($t_1 = 0.5$ years) and 10 months ($t_2 = \frac{10}{12} \approx 0.8333$ years). Each semiannual fixed payment is 2\% of the notional amount. 

The present value of these payments is:

\[
PV_{\text{fixed}} = 2 \times e^{-0.027 \times 0.5} + 102 \times e^{-0.027 \times 0.8333}
\]

where the first term represents the payment at $t_1$ and the second term includes the notional repayment ($100$ million) plus the final fixed payment.

\subsection*{Step 2: Present Value of the Floating Leg}
The floating rate payment for the first 6 months is based on the LIBOR rate of 2.4\%, which was set two months ago. Therefore, the floating payment at $t_1$ (in 6 months) is:

\[
\text{Floating Payment at } t_1 = 100 \times 2.4\% = 2.4 \text{ million USD}
\]

For the next floating rate payment at $t_2$, the forward rate is 3\%. Therefore, the floating payment at $t_2$ (in 10 months) is:

\[
\text{Floating Payment at } t_2 = 100 \times 3\% = 3 \text{ million USD}
\]

The present value of the floating leg is:

\[
PV_{\text{floating}} = 2.4 \times e^{-0.027 \times 0.5} + 103 \times e^{-0.027 \times 0.8333}
\]

where the $103$ million includes the notional plus the final floating payment.

\subsection*{Step 3: Net Value of the Swap}

The value of the swap to the party paying fixed is the difference between the present values of the floating and fixed legs:

\[
\text{Value to Fixed Payer} = PV_{\text{floating}} - PV_{\text{fixed}}
\]

The value of the swap to the party paying floating is the negative of the above value.


\end{solution}


    \newpage
    
    \question A currency swap has a remaining life of 15 months. It involves exchanging interest at
10\% on £20 million for interest at 6\% on \$30 million once a year. The term structure of
risk-free interest rates in the United Kingdom is flat at 7\% and the term structure of
riskfree interest rates in the United States is flat at 4\% (both with annual compounding).
The current exchange rate (dollars per pound sterling) is 1.5500. What is the value of the
swap to the party paying sterling? What is the value of the swap to the party paying
dollars? (Hull 7.14)

    \begin{solution}
    \subsection*{Step 1: Value of Sterling Payments}

The party paying GBP will pay 10\% of £20 million at the end of the first year, i.e.,

\[
\text{GBP Payment} = 0.10 \times 20 = 2 \, \text{million GBP}
\]

Additionally, the notional £20 million will be paid back at the end of 15 months. The present value of these payments, discounted at the UK risk-free rate of 7\%, is:

\[
PV_{\text{GBP}} = 2 \times e^{-0.07 \times 1} + 20 \times e^{-0.07 \times \frac{15}{12}}
\]

\subsection*{Step 2: Value of Dollar Payments}

The party paying USD will pay 6\% of \$30 million at the end of the first year, i.e.,

\[
\text{USD Payment} = 0.06 \times 30 = 1.8 \, \text{million USD}
\]

Additionally, the notional \$30 million will be paid back at the end of 15 months. The present value of these payments, discounted at the US risk-free rate of 4\%, is:

\[
PV_{\text{USD}} = 1.8 \times e^{-0.04 \times 1} + 30 \times e^{-0.04 \times \frac{15}{12}}
\]

\subsection*{Step 3: Convert GBP Payments to USD}

The present value of the GBP payments in USD, using the exchange rate of 1.5500 USD/GBP, is:

\[
PV_{\text{GBP in USD}} = 1.5500 \times PV_{\text{GBP}}
\]

\subsection*{Step 4: Value of the Swap to Each Party}

The value of the swap to the party paying sterling is:

\[
\text{Value to GBP Payer} = PV_{\text{USD}} - PV_{\text{GBP in USD}}
\]

The value of the swap to the party paying dollars is:

\[
\text{Value to USD Payer} = -(\text{Value to GBP Payer})
\]


\end{solution}


    \pagebreak %Not necessary
    
    \question A financial institution has entered into an interest rate swap with company X. Under
the terms of the swap, it receives 4\% per annum and pays six-month LIBOR on a
principal of \$10 million for five years. Payments are made every six months. Suppose
that company X defaults on the sixth payment date (end of year 3) when six-month
forward LIBOR rates for all maturities are 2\% per annum. What is the loss to the
financial institution? Assume that six-month LIBOR was 3\% per annum halfway through
year 3 and that at the time of the default all OIS rates are 1.8\% per annum. OIS rates are
expressed with continuous compounding; other rates are expressed with semiannual
compounding. (Hull 7.19)

    \begin{solution}
    \subsection*{Step 1: Fixed Payments After Default}
The financial institution is receiving 4\% per annum, or 2\% every six months on \$10 million. The fixed payment is:

\[
\text{Fixed Payment} = 0.02 \times 10 = 0.2 \, \text{million USD per period}
\]

Since four payments remain after the default, the total fixed payments are:

\[
\text{Total Fixed Payments} = 0.2 \times 4 = 0.8 \, \text{million USD}
\]

\subsection*{Step 2: Floating Payments After Default}
The floating payments are based on six-month LIBOR, which is 2\% per annum, or 1\% for six months. The floating payment is:

\[
\text{Floating Payment} = 0.01 \times 10 = 0.1 \, \text{million USD per period}
\]

For four periods, the total floating payments are:

\[
\text{Total Floating Payments} = 0.1 \times 4 = 0.4 \, \text{million USD}
\]

\subsection*{Step 3: Net Cash Flow}
The net cash flow per period is the difference between the fixed payment and the floating payment:

\[
\text{Net Cash Flow} = 0.2 - 0.1 = 0.1 \, \text{million USD per period}
\]

Over four periods, the total net cash flow is:

\[
\text{Total Net Cash Flow} = 0.1 \times 4 = 0.4 \, \text{million USD}
\]

\subsection*{Step 4: Present Value of Lost Cash Flows}
The cash flows must be discounted using the OIS rate of 1.8\%, compounded continuously. The discount factor for a payment at time \(t\) is:

\[
\text{Discount Factor} = e^{-0.018 \times t}
\]

The present value of the total lost cash flows is:

\[
PV_{\text{total}} = 0.1 \times e^{-0.018 \times 0.5} + 0.1 \times e^{-0.018 \times 1.0} + 0.1 \times e^{-0.018 \times 1.5} + 0.1 \times e^{-0.018 \times 2.0}
\]

\subsection*{Step 5: Total Loss to the Financial Institution}
The total loss is the sum of the present values of the net cash flows:

\[
\text{Total Loss} = PV_{\text{total}}
\]


\end{solution}


    \newpage
    \question OIS rates are 3.4\% for all maturities. What is the value of an OIS swap with two years to maturity where 3\% is received and the floating reference rate is paid? Assume annual compounding, annual payments, and a \$100 million principal.

    \begin{solution}
\subsection*{Step 1: Present Value of Fixed Payments}
The fixed payments are 3\% annually on a \$100 million notional. The fixed payment is:

\[
\text{Fixed Payment} = 0.03 \times 100 = 3 \, \text{million USD annually}
\]

The present value of these fixed payments over two years, discounted at the OIS rate of 3.4\%, is:

\[
PV_{\text{fixed}} = 3 \times \frac{1}{(1 + 0.034)^1} + 103 \times \frac{1}{(1 + 0.034)^2}
\]

\[
PV_{\text{fixed}} = 3 \times 0.967 + 103 \times 0.935 = 2.901 + 96.305 = 99.206 \, \text{million USD}
\]

\subsection*{Step 2: Present Value of Floating Payments}
The floating rate is 3.4\% annually on a \$100 million notional. The floating payment is:

\[
\text{Floating Payment} = 0.034 \times 100 = 3.4 \, \text{million USD annually}
\]

The present value of these floating payments over two years, discounted at the OIS rate of 3.4\%, is:

\[
PV_{\text{floating}} = 3.4 \times \frac{1}{(1 + 0.034)^1} + 103.4 \times \frac{1}{(1 + 0.034)^2}
\]

\[
PV_{\text{floating}} = 3.4 \times 0.967 + 103.4 \times 0.935 = 3.288 + 96.679 = 99.967 \, \text{million USD}
\]

\subsection*{Step 3: Net Value of the Swap}
The net value of the swap is the difference between the present values of the fixed and floating legs:

\[
\text{Swap Value} = PV_{\text{fixed}} - PV_{\text{floating}} = 99.206 - 99.967 = -0.761 \, \text{million USD}
\]

\end{solution}
    
\end{questions}
\end{document}